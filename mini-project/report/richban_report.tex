
%% bare_jrnl.tex
%% V1.4a
%% 2014/09/17
%% by Michael Shell
%% see http://www.michaelshell.org/
%% for current contact information.
%%
%% This is a skeleton file demonstrating the use of IEEEtran.cls
%% (requires IEEEtran.cls version 1.8a or later) with an IEEE
%% journal paper.
%%
%% Support sites:
%% http://www.michaelshell.org/tex/ieeetran/
%% http://www.ctan.org/tex-archive/macros/latex/contrib/IEEEtran/
%% and
%% http://www.ieee.org/

%%*************************************************************************
%% Legal Notice:
%% This code is offered as-is without any warranty either expressed or
%% implied; without even the implied warranty of MERCHANTABILITY or
%% FITNESS FOR A PARTICULAR PURPOSE! 
%% User assumes all risk.
%% In no event shall IEEE or any contributor to this code be liable for
%% any damages or losses, including, but not limited to, incidental,
%% consequential, or any other damages, resulting from the use or misuse
%% of any information contained here.
%%
%% All comments are the opinions of their respective authors and are not
%% necessarily endorsed by the IEEE.
%%
%% This work is distributed under the LaTeX Project Public License (LPPL)
%% ( http://www.latex-project.org/ ) version 1.3, and may be freely used,
%% distributed and modified. A copy of the LPPL, version 1.3, is included
%% in the base LaTeX documentation of all distributions of LaTeX released
%% 2003/12/01 or later.
%% Retain all contribution notices and credits.
%% ** Modified files should be clearly indicated as such, including  **
%% ** renaming them and changing author support contact information. **
%%
%% File list of work: IEEEtran.cls, IEEEtran_HOWTO.pdf, bare_adv.tex,
%%                    bare_conf.tex, bare_jrnl.tex, bare_conf_compsoc.tex,
%%                    bare_jrnl_compsoc.tex, bare_jrnl_transmag.tex
%%*************************************************************************


% *** Authors should verify (and, if needed, correct) their LaTeX system  ***
% *** with the testflow diagnostic prior to trusting their LaTeX platform ***
% *** with production work. IEEE's font choices and paper sizes can       ***
% *** trigger bugs that do not appear when using other class files.       ***                          ***
% The testflow support page is at:
% http://www.michaelshell.org/tex/testflow/



\documentclass[journal]{IEEEtran}
%
% If IEEEtran.cls has not been installed into the LaTeX system files,
% manually specify the path to it like:
% \documentclass[journal]{../sty/IEEEtran}





% Some very useful LaTeX packages include:
% (uncomment the ones you want to load)


% *** MISC UTILITY PACKAGES ***
%
%\usepackage{ifpdf}
% Heiko Oberdiek's ifpdf.sty is very useful if you need conditional
% compilation based on whether the output is pdf or dvi.
% usage:
% \ifpdf
%   % pdf code
% \else
%   % dvi code
% \fi
% The latest version of ifpdf.sty can be obtained from:
% http://www.ctan.org/tex-archive/macros/latex/contrib/oberdiek/
% Also, note that IEEEtran.cls V1.7 and later provides a builtin
% \ifCLASSINFOpdf conditional that works the same way.
% When switching from latex to pdflatex and vice-versa, the compiler may
% have to be run twice to clear warning/error messages.






% *** CITATION PACKAGES ***
%
%\usepackage{cite}
% cite.sty was written by Donald Arseneau
% V1.6 and later of IEEEtran pre-defines the format of the cite.sty package
% \cite{} output to follow that of IEEE. Loading the cite package will
% result in citation numbers being automatically sorted and properly
% "compressed/ranged". e.g., [1], [9], [2], [7], [5], [6] without using
% cite.sty will become [1], [2], [5]--[7], [9] using cite.sty. cite.sty's
% \cite will automatically add leading space, if needed. Use cite.sty's
% noadjust option (cite.sty V3.8 and later) if you want to turn this off
% such as if a citation ever needs to be enclosed in parenthesis.
% cite.sty is already installed on most LaTeX systems. Be sure and use
% version 5.0 (2009-03-20) and later if using hyperref.sty.
% The latest version can be obtained at:
% http://www.ctan.org/tex-archive/macros/latex/contrib/cite/
% The documentation is contained in the cite.sty file itself.






% *** GRAPHICS RELATED PACKAGES ***
%
\ifCLASSINFOpdf
\usepackage[pdftex]{graphicx}
  % declare the path(s) where your graphic files are
%Path in Unix-like (Linux, OsX) format
\graphicspath{ {img/}}
  % and their extensions so you won't have to specify these with
\includegraphics
\DeclareGraphicsExtensions{.pdf,.jpeg,.png}
\else
  % or other class option (dvipsone, dvipdf, if not using dvips). graphicx
  % will default to the driver specified in the system graphics.cfg if no
  % driver is specified.
  % \usepackage[dvips]{graphicx}
  % declare the path(s) where your graphic files are
  % \graphicspath{{../eps/}}
  % and their extensions so you won't have to specify these with
  % every instance of \includegraphics
  % \DeclareGraphicsExtensions{.eps}
\fi
% graphicx was written by David Carlisle and Sebastian Rahtz. It is
% required if you want graphics, photos, etc. graphicx.sty is already
% installed on most LaTeX systems. The latest version and documentation
% can be obtained at: 
% http://www.ctan.org/tex-archive/macros/latex/required/graphics/
% Another good source of documentation is "Using Imported Graphics in
% LaTeX2e" by Keith Reckdahl which can be found at:
% http://www.ctan.org/tex-archive/info/epslatex/
%
% latex, and pdflatex in dvi mode, support graphics in encapsulated
% postscript (.eps) format. pdflatex in pdf mode supports graphics
% in .pdf, .jpeg, .png and .mps (metapost) formats. Users should ensure
% that all non-photo figures use a vector format (.eps, .pdf, .mps) and
% not a bitmapped formats (.jpeg, .png). IEEE frowns on bitmapped formats
% which can result in "jaggedy"/blurry rendering of lines and letters as
% well as large increases in file sizes.
%
% You can find documentation about the pdfTeX application at:
% http://www.tug.org/applications/pdftex



\usepackage{listings}
\usepackage{listings}
\usepackage{color}
 
\definecolor{codegreen}{rgb}{0,0.6,0}
\definecolor{codegray}{rgb}{0.5,0.5,0.5}
\definecolor{codepurple}{rgb}{0.58,0,0.82}
\definecolor{backcolour}{rgb}{0.95,0.95,0.92}
 
\lstdefinestyle{mystyle}{
    backgroundcolor=\color{backcolour},   
    commentstyle=\color{codegreen},
    keywordstyle=\color{magenta},
    numberstyle=\tiny\color{codegray},
    stringstyle=\color{codepurple},
    basicstyle=\footnotesize,
    breakatwhitespace=false,         
    breaklines=true,                 
    captionpos=b,                    
    keepspaces=true,                 
    numbers=left,                    
    numbersep=5pt,                  
    showspaces=false,                
    showstringspaces=false,
    showtabs=false,                  
    tabsize=2
}
 
\lstset{style=mystyle}

% *** MATH PACKAGES ***
%
%\usepackage[cmex10]{amsmath}
% A popular package from the American Mathematical Society that provides
% many useful and powerful commands for dealing with mathematics. If using
% it, be sure to load this package with the cmex10 option to ensure that
% only type 1 fonts will utilized at all point sizes. Without this option,
% it is possible that some math symbols, particularly those within
% footnotes, will be rendered in bitmap form which will result in a
% document that can not be IEEE Xplore compliant!
%
% Also, note that the amsmath package sets \interdisplaylinepenalty to 10000
% thus preventing page breaks from occurring within multiline equations. Use:
%\interdisplaylinepenalty=2500
% after loading amsmath to restore such page breaks as IEEEtran.cls normally
% does. amsmath.sty is already installed on most LaTeX systems. The latest
% version and documentation can be obtained at:
% http://www.ctan.org/tex-archive/macros/latex/required/amslatex/math/





% *** SPECIALIZED LIST PACKAGES ***
%
%\usepackage{algorithmic}
% algorithmic.sty was written by Peter Williams and Rogerio Brito.
% This package provides an algorithmic environment fo describing algorithms.
% You can use the algorithmic environment in-text or within a figure
% environment to provide for a floating algorithm. Do NOT use the algorithm
% floating environment provided by algorithm.sty (by the same authors) or
% algorithm2e.sty (by Christophe Fiorio) as IEEE does not use dedicated
% algorithm float types and packages that provide these will not provide
% correct IEEE style captions. The latest version and documentation of
% algorithmic.sty can be obtained at:
% http://www.ctan.org/tex-archive/macros/latex/contrib/algorithms/
% There is also a support site at:
% http://algorithms.berlios.de/index.html
% Also of interest may be the (relatively newer and more customizable)
% algorithmicx.sty package by Szasz Janos:
% http://www.ctan.org/tex-archive/macros/latex/contrib/algorithmicx/




% *** ALIGNMENT PACKAGES ***
%
%\usepackage{array}
% Frank Mittelbach's and David Carlisle's array.sty patches and improves
% the standard LaTeX2e array and tabular environments to provide better
% appearance and additional user controls. As the default LaTeX2e table
% generation code is lacking to the point of almost being broken with
% respect to the quality of the end results, all users are strongly
% advised to use an enhanced (at the very least that provided by array.sty)
% set of table tools. array.sty is already installed on most systems. The
% latest version and documentation can be obtained at:
% http://www.ctan.org/tex-archive/macros/latex/required/tools/


% IEEEtran contains the IEEEeqnarray family of commands that can be used to
% generate multiline equations as well as matrices, tables, etc., of high
% quality.




% *** SUBFIGURE PACKAGES ***
%\ifCLASSOPTIONcompsoc
%  \usepackage[caption=false,font=normalsize,labelfont=sf,textfont=sf]{subfig}
%\else
%  \usepackage[caption=false,font=footnotesize]{subfig}
%\fi
% subfig.sty, written by Steven Douglas Cochran, is the modern replacement
% for subfigure.sty, the latter of which is no longer maintained and is
% incompatible with some LaTeX packages including fixltx2e. However,
% subfig.sty requires and automatically loads Axel Sommerfeldt's caption.sty
% which will override IEEEtran.cls' handling of captions and this will result
% in non-IEEE style figure/table captions. To prevent this problem, be sure
% and invoke subfig.sty's "caption=false" package option (available since
% subfig.sty version 1.3, 2005/06/28) as this is will preserve IEEEtran.cls
% handling of captions.
% Note that the Computer Society format requires a larger sans serif font
% than the serif footnote size font used in traditional IEEE formatting
% and thus the need to invoke different subfig.sty package options depending
% on whether compsoc mode has been enabled.
%
% The latest version and documentation of subfig.sty can be obtained at:
% http://www.ctan.org/tex-archive/macros/latex/contrib/subfig/




% *** FLOAT PACKAGES ***
%
%\usepackage{fixltx2e}
% fixltx2e, the successor to the earlier fix2col.sty, was written by
% Frank Mittelbach and David Carlisle. This package corrects a few problems
% in the LaTeX2e kernel, the most notable of which is that in current
% LaTeX2e releases, the ordering of single and double column floats is not
% guaranteed to be preserved. Thus, an unpatched LaTeX2e can allow a
% single column figure to be placed prior to an earlier double column
% figure. The latest version and documentation can be found at:
% http://www.ctan.org/tex-archive/macros/latex/base/


%\usepackage{stfloats}
% stfloats.sty was written by Sigitas Tolusis. This package gives LaTeX2e
% the ability to do double column floats at the bottom of the page as well
% as the top. (e.g., "\begin{figure*}[!b]" is not normally possible in
% LaTeX2e). It also provides a command:
%\fnbelowfloat
% to enable the placement of footnotes below bottom floats (the standard
% LaTeX2e kernel puts them above bottom floats). This is an invasive package
% which rewrites many portions of the LaTeX2e float routines. It may not work
% with other packages that modify the LaTeX2e float routines. The latest
% version and documentation can be obtained at:
% http://www.ctan.org/tex-archive/macros/latex/contrib/sttools/
% Do not use the stfloats baselinefloat ability as IEEE does not allow
% \baselineskip to stretch. Authors submitting work to the IEEE should note
% that IEEE rarely uses double column equations and that authors should try
% to avoid such use. Do not be tempted to use the cuted.sty or midfloat.sty
% packages (also by Sigitas Tolusis) as IEEE does not format its papers in
% such ways.
% Do not attempt to use stfloats with fixltx2e as they are incompatible.
% Instead, use Morten Hogholm'a dblfloatfix which combines the features
% of both fixltx2e and stfloats:
%
% \usepackage{dblfloatfix}
% The latest version can be found at:
% http://www.ctan.org/tex-archive/macros/latex/contrib/dblfloatfix/




%\ifCLASSOPTIONcaptionsoff
%  \usepackage[nomarkers]{endfloat}
% \let\MYoriglatexcaption\caption
% \renewcommand{\caption}[2][\relax]{\MYoriglatexcaption[#2]{#2}}
%\fi
% endfloat.sty was written by James Darrell McCauley, Jeff Goldberg and 
% Axel Sommerfeldt. This package may be useful when used in conjunction with 
% IEEEtran.cls'  captionsoff option. Some IEEE journals/societies require that
% submissions have lists of figures/tables at the end of the paper and that
% figures/tables without any captions are placed on a page by themselves at
% the end of the document. If needed, the draftcls IEEEtran class option or
% \CLASSINPUTbaselinestretch interface can be used to increase the line
% spacing as well. Be sure and use the nomarkers option of endfloat to
% prevent endfloat from "marking" where the figures would have been placed
% in the text. The two hack lines of code above are a slight modification of
% that suggested by in the endfloat docs (section 8.4.1) to ensure that
% the full captions always appear in the list of figures/tables - even if
% the user used the short optional argument of \caption[]{}.
% IEEE papers do not typically make use of \caption[]'s optional argument,
% so this should not be an issue. A similar trick can be used to disable
% captions of packages such as subfig.sty that lack options to turn off
% the subcaptions:
% For subfig.sty:
% \let\MYorigsubfloat\subfloat
% \renewcommand{\subfloat}[2][\relax]{\MYorigsubfloat[]{#2}}
% However, the above trick will not work if both optional arguments of
% the \subfloat command are used. Furthermore, there needs to be a
% description of each subfigure *somewhere* and endfloat does not add
% subfigure captions to its list of figures. Thus, the best approach is to
% avoid the use of subfigure captions (many IEEE journals avoid them anyway)
% and instead reference/explain all the subfigures within the main caption.
% The latest version of endfloat.sty and its documentation can obtained at:
% http://www.ctan.org/tex-archive/macros/latex/contrib/endfloat/
%
% The IEEEtran \ifCLASSOPTIONcaptionsoff conditional can also be used
% later in the document, say, to conditionally put the References on a 
% page by themselves.




% *** PDF, URL AND HYPERLINK PACKAGES ***
%
%\usepackage{url}
% url.sty was written by Donald Arseneau. It provides better support for
% handling and breaking URLs. url.sty is already installed on most LaTeX
% systems. The latest version and documentation can be obtained at:
% http://www.ctan.org/tex-archive/macros/latex/contrib/url/
% Basically, \url{my_url_here}.




% *** Do not adjust lengths that control margins, column widths, etc. ***
% *** Do not use packages that alter fonts (such as pslatex).         ***
% There should be no need to do such things with IEEEtran.cls V1.6 and later.
% (Unless specifically asked to do so by the journal or conference you plan
% to submit to, of course. )


% correct bad hyphenation here
\hyphenation{op-tical net-works semi-conduc-tor}


\begin{document}
%
% paper title
% Titles are generally capitalized except for words such as a, an, and, as,
% at, but, by, for, in, nor, of, on, or, the, to and up, which are usually
% not capitalized unless they are the first or last word of the title.
% Linebreaks \\ can be used within to get better formatting as desired.
% Do not put math or special symbols in the title.
\title{Data Mining ITU 2017 Spring - Individual Assignment}
%
%
% author names and IEEE memberships
% note positions of commas and nonbreaking spaces ( ~ ) LaTeX will not break
% a structure at a ~ so this keeps an author's name from being broken across
% two lines.
% use \thanks{} to gain access to the first footnote area
% a separate \thanks must be used for each paragraph as LaTeX2e's \thanks
% was not built to handle multiple paragraphs
%

\author{Richard~Banyi,~\IEEEmembership{Student,~ITU}% <-this % stops a space
\thanks{P.  González de Prado Salas is with the Department
of Advanced Programming, IT University of Copenhagen, Denmark}% <-this % stops a space
\thanks{S. Risi Head of Data Mining Cource, IT University of Copenhagen.}% <-this % stops a space
\thanks{Manuscript received May 26, 2017}}

% note the % following the last \IEEEmembership and also \thanks - 
% these prevent an unwanted space from occurring between the last author name
% and the end of the author line. i.e., if you had this:
% 
% \author{....lastname \thanks{...} \thanks{...} }
%                     ^------------^------------^----Do not want these spaces!
%
% a space would be appended to the last name and could cause every name on that
% line to be shifted left slightly. This is one of those "LaTeX things". For
% instance, "\textbf{A} \textbf{B}" will typeset as "A B" not "AB". To get
% "AB" then you have to do: "\textbf{A}\textbf{B}"
% \thanks is no different in this regard, so shield the last } of each \thanks
% that ends a line with a % and do not let a space in before the next \thanks.
% Spaces after \IEEEmembership other than the last one are OK (and needed) as
% you are supposed to have spaces between the names. For what it is worth,
% this is a minor point as most people would not even notice if the said evil
% space somehow managed to creep in.



% The paper headers
\markboth{Report of Data Mining Individual Assignment, No.~1, May~2017}%
{Shell \MakeLowercase{\textit{et al.}}: Data Mining ITU 2017 Spring - Individual Assignment}
% The only time the second header will appear is for the odd numbered pages
% after the title page when using the twoside option.
% 
% *** Note that you probably will NOT want to include the author's ***
% *** name in the headers of peer review papers.                   ***
% You can use \ifCLASSOPTIONpeerreview for conditional compilation here if
% you desire.




% If you want to put a publisher's ID mark on the page you can do it like
% this:
%\IEEEpubid{0000--0000/00\$00.00~\copyright~2014 IEEE}
% Remember, if you use this you must call \IEEEpubidadjcol in the second
% column for its text to clear the IEEEpubid mark.



% use for special paper notices
%\IEEEspecialpapernotice{(Invited Paper)}




% make the title area
\maketitle

% As a general rule, do not put math, special symbols or citations
% in the abstract or keywords.
%\begin{abstract}
%The abstract goes here.
%\end{abstract}

% Note that keywords are not normally used for peerreview papers.
\begin{IEEEkeywords}
Data-mining, preprocessing, KNN classification, K-Means Clustering, Apriori - frequent pattern mining.
\end{IEEEkeywords}






% For peer review papers, you can put extra information on the cover
% page as needed:
% \ifCLASSOPTIONpeerreview
% \begin{center} \bfseries EDICS Category: 3-BBND \end{center}
% \fi
%
% For peerreview papers, this IEEEtran command inserts a page break and
% creates the second title. It will be ignored for other modes.
\IEEEpeerreviewmaketitle



\section{Data Preprocessing}

\IEEEPARstart{T}{h}e first part of the preprocessing was to getting know the dataset and explore if the data quality satisfy the requirements of the intented use. Firstly I have loaded the dataset and examined the dimensionality and the accuracy, completeness, consistency of the data. The input dataset consisted of 67 rows x 45 colums. After the dataset was identified I have created several methods for data cleaning for filling in missing values, filtering out numbers with regex, convert to float data types, replacing inconsistencies.
 
% You must have at least 2 lines in the paragraph with the drop letter
% (should never be an issue)

% \hfill mds
 
% \hfill September 17, 2014

\subsection{Dimensionality Reduction}
After the data was cleaned, I have picked the features that I have decided to used for implementing KNN for classification, K\-Means for Clustering and Apriori \- frequent pattern mining. The subset of features I have used are: \textit{age, shoe\_size, height, language, gender.} Most of the features like \textit{age, shoe\_size, height} are discreate features, I have used descriptive statistics to see the distribution of these features. The other 2 features were nominal \textit{gender} and \textit{language}

\begin{table}[ht]
%% increase table row spacing, adjust to taste
% \renewcommand{\arraystretch}{1.3}
% if using array.sty, it might be a good idea to tweak the value of
% \extrarowheight as needed to properly center the text within the cells
\caption{Descriptive Statistic}
\centering
%% Some packages, such as MDW tools, offer better commands for making tables
%% than the plain LaTeX2e tabular which is used here.
\begin{tabular}{|c||c|c|c|}
\hline
& age & shoe\_size & height \\
\hline
count & 67 & 67 & 67 \\ 
\hline
mean & 40.701493 & 41.537313 & 175.298507 \\
\hline
std & 118.907784 & 5.915640 & 25.702856 \\ 
\hline
min & 22.000000 & 2.000000 & 34.000000 \\ 
\hline
25\% & 24.000000 & 40.750000 & 172.000000 \\ 
\hline
50\%  & 25.000000 & 42.500000 & 180.000000  \\
\hline
75\%  & 28.000000 & 44.250000 & 186.500000  \\
\hline
max  & 999.000000 & 49.000000 & 205.000000  \\
\hline
\end{tabular}
\end{table}

\subsection{Data Transformation}
Through the use of descriptive statistics, it was clear that the discreate features need to be normalized. Therefore I have used Z-score normalization to ensure that all the features are rescaled and contribute equally, which was crucial in measuring Euclidean distance. 


\begin{figure}[ht]
\centering
\includegraphics[width=2.5in]{z_score}
\caption{Z-score Normalization}
\end{figure}

Alternaly I have also used Min-Max scalling, in this case the data is scaled to a fixed range bettween 0-1.

\begin{figure}[ht]
\centering
\includegraphics[width=2.5in]{minmax_scaling}
\caption{Min-Max Scalling}
\end{figure}

In order to implement apriori - frequent pattern mining I had to split the data so that every programming language will be atomic, e.g. in each column is one value.

\begin{lstlisting}
[[r, java],
 [java, css, html],
 [javascript, java],
 [java], ... ]
\end{lstlisting}

\section{Clustering}

I have chosen K-means as the clustering algorithm. My goal was to find out the clusters on discreate features \textit{age, height and shoe size}. The algorithm first select random k objects (vectors) from the dataset as initial clusters centers. Than for each object (vector) from the dataset the Euclidean distance is computed between that object and the cluster centroids and it's assigned to the cluster which is the most similiar (shortest distance). Afterwards each cluster centers are recomputed and all the objects are then reassigned using the updated cluster centers. And the iteration continues until the centroids stabilize. The algorithm calculate how much the centroids moved in each iteration and compare from the previous.
\subsection{Results}

The graph below shows the data points and the cluster center points marked as X in 3 dimensional space. In order to choose the appropriate K I have measured the cluster quality by the method called \textit{within-cluster variaton}, e.g. sum of squared error. I've computed the summed squared distances for each cluster, which is the inertia and took the average of all the clusters. As the chart below shows, as I have increased K the inertia  went down - the lower intertia is the data points are closer to the centre points, therefore more clusters means data points get represented better. In conclusion, the best value for K was the point where by increased K it didn't reduce the inertia much more, therefore \textit{k=5}.

\begin{figure}[ht]
\centering
\includegraphics[width=2.5in]{cluster_variance}
\caption{Average within-cluster variation}
\end{figure}

\begin{figure}[ht]
\centering
\includegraphics[width=2.5in]{clustering}
\caption{Cluster Centroids}
\end{figure}


\section{Classification}

For the supervised learning I have choosed to implement K-Nearest Neighbours. Again I have used the same dataset as for the the clustering, therefore the question I have tried to find answer is \textit{What is the person's gender based on their age, heigh and shoe size?}

\subsection{Results}

I have used a standart ratio of spliting the data (2/3 for training and 1/3 for testing). I have gained accuracy mostly around 80\% - 90\% correct predicted class for the gender. I suppose this high accuracy is the result that the dataset is unbalanced. There is total of 55 males and 9 females classifiers. The line chart below shows how the accuracy differ when I have increased \textit{k} for the following spliting: Train: 43 and Test: 20.

\begin{figure}[ht]
\centering
\includegraphics[width=2.5in]{classification}
\caption{Classification}
\end{figure}

Also the accuracy greatly varied accross different ratio of training and testing. The chart below shows different accuracy obtained for different split of data (train and test) and \textit{k = 5.}

\begin{figure}[ht]
\centering
\includegraphics[width=2.5in]{diff_split}
\caption{KNN Classificatio with different split}
\end{figure}

\section{Apriori}

Lastly, for the frequent pattern mining I have used Apriori Algorithm in order to found out which programming languages are most likely group together within ITU students. \textit{What itemset of programming languages do itu students know?}. 

\subsection{Results}

The implementation consisted of 2 parts, first I have found the all the frequent itemsets which meet the minimum support level \textit{min\_sup=0.2}.

\begin{lstlisting}
[[({'c#'}),
  ({'c'}),
  ({'python'}),
  ({'javascript'}),
  ({'java'}),
  ({'f#'}),
  ({'c++'})],
 [({'c', 'java'}),
  ({'c#', 'java'}),
  ({'c++', 'java'}),
  ({'c#', 'python'}),
  ({'f#', 'java'}),
  ({'c#', 'c++'}),
  ({'java', 'javascript'}),
  ({'java', 'python'}),
  ({'c#', 'javascript'})],
 [({'c#', 'java', 'javascript'}),
  ({'c#', 'c++', 'java'}),
  ({'c#', 'java', 'python'})]
\end{lstlisting}
 
 Then I have generated strong assocation rules from that frequent itemset which satisfied the minimum support and minimum confidence \textit{min\_conf=0.7}.
 
 \begin{lstlisting}
[(({'c'}), ({'java'}), 1.0),
 (({'c#'}), ({'java'}), 0.9655172413793103),
 (({'c++'}), ({'java'}), 0.95),
 (({'f#'}), ({'java'}), 1.0),
 (({'c++'}), ({'c#'}), 0.9000000000000001),
 (({'javascript'}), ({'java'}), 0.95),
 (({'python'}), ({'java'}), 0.9047619047619047),
 (({'c++'}), ({'c#', 'java'}), 0.8500000000000001)]
 \end{lstlisting}
 
The rules above show up in at least 20\% off all the transactions. Therefore 90\% of students who knows python also knows java programming language.

%\begin{lstlisting}
%MLPClassifier(activation='relu', alpha=1e-05, batch_size='auto', beta_1=0.9,
%       beta_2=0.999, early_stopping=False, epsilon=1e-08,
%       hidden_layer_sizes=(15, 15, 15), learning_rate='constant',
%       learning_rate_init=0.001, max_iter=10000, momentum=0.9,
%       nesterovs_momentum=True, power_t=0.5, random_state=1, shuffle=True,
%       solver='lbfgs', tol=0.0001, validation_fraction=0.1, verbose=False,
%       warm_start=False)
%\end{lstlisting}

% Can use something like this to put references on a page
% by themselves when using endfloat and the captionsoff option.
\ifCLASSOPTIONcaptionsoff
  \newpage
\fi

% that's all folks
\end{document}


